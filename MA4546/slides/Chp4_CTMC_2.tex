\documentclass[english,10pt,handout]{beamer}
%\documentclass[english,11pt]{beamer}

\usepackage[clock]{ifsym}
%\usepackage{clock}






 
%\usepackage{mathptmx}
%\renewcommand{\sfdefault}{lmss}
\usepackage[T1]{fontenc}
%\usepackage[latin9]{inputenc}
\usepackage[utf8]{inputenc}

\synctex=-1

\usefonttheme{professionalfonts}

%\setbeamertemplate{navigation symbols}{}
%\setbeamertemplate{caption}[numbered]


\useinnertheme{rectangles}
%http://tex.stackexchange.com/questions/11168/change-bullet-style-formatting-in-beamer

 \AtBeginDocument{
  \addtolength\abovedisplayskip{-0.4\baselineskip}%
  \addtolength\belowdisplayskip{-0.4\baselineskip}%
}%change the space between text lines and the math formula


\usepackage{pifont}
%Postscript ZipfDingbats font
%the command \ding{number}, will print the specified symbol

\usepackage{fontawesome}
%icon package
\DeclareFontFamily{U}{FontAwesomeOne}{}
\DeclareFontShape{U}{FontAwesomeOne}{m}{n}{<-> FontAwesome--fontawesomeone}{}
\DeclareRobustCommand\FAone{\fontencoding{U}\fontfamily{FontAwesomeOne}\fontseries{m}\fontshape{n}\selectfont}
\DeclareFontFamily{U}{FontAwesomeTwo}{}
\DeclareFontShape{U}{FontAwesomeTwo}{m}{n}{<-> FontAwesome--fontawesometwo}{}
\DeclareRobustCommand\FAtwo{\fontencoding{U}\fontfamily{FontAwesomeTwo}\fontseries{m}\fontshape{n}\selectfont}
\DeclareFontFamily{U}{FontAwesomeThree}{}
\DeclareFontShape{U}{FontAwesomeThree}{m}{n}{<-> FontAwesome--fontawesomethree}{}
\DeclareRobustCommand\FAthree{\fontencoding{U}\fontfamily{FontAwesomeThree}\fontseries{m}\fontshape{n}\selectfont}

%ftp://ftp.dante.de/tex-archive/fonts/fontawesome/doc/fontawesome.pdf
%http://tug.ctan.org/info/symbols/comprehensive/symbols-a4.pdf


\usepackage{amsmath,amssymb,amsfonts,bm,mathrsfs,mathtools}

\usepackage{tikzsymbols}
%\usepackage[tikz]{bclogo}



\usepackage{perpage}
\MakePerPage{footnote} %reset for each page
%\renewcommand{\thefootnote}{\fnsymbol{footnote}} %use symbol, limit less than 9 symbols



%%%% HIGHTLIGHT  and annotation &=%%%%%%%%
\usepackage{color,xcolor}
 \usepackage{todonotes}

\usepackage[normalem]{ulem}

\usepackage[many]{tcolorbox}

\tcbset{fonttitle=\scriptsize}
\tcbset{highlight math style={enhanced,
  colframe=red!40!black,colback=yellow!20!white,arc=2pt,boxrule=.2pt,
  }}
  \newtcbox{\otherbox}[1][]{nobeforeafter,math upper,tcbox raise base,
enhanced,frame hidden,boxrule=0pt,interior style={top color=green!10!white,
bottom color=green!10!white,middle color=green!50!yellow},
fuzzy halo=1pt with green,#1}
%%\tcbhighmath{math here}
%% \otherbox{math here}



%%%%% HIGHLIGHT %%%%%%
\newcommand{\hb}[1]{{\color{blue}{#1}}}
%\noindent\rule{\textwidth}{.5pt}

%:
\usepackage{soul}

\newcommand\hcancel[2][black]{\setbox0=\hbox{$#2$}%
\rlap{\raisebox{.45\ht0}{\textcolor{#1}{\rule{\wd0}{1pt}}}}#2}
%cross to delete

\newcommand{\mcb}[2]{\colorbox{#1}{$\displaystyle #2$}}
%highlight math

\newcommand{\hlfancy}[2]{\sethlcolor{#1}\hl{#2}}
%specified color , for\hl

\newcommand\myhl{\bgroup\markoverwith
  {\textcolor{yellow}{\rule[-.5ex]{2pt}{2.5ex}}}\ULon}



\mode<presentation>{ \usetheme{boxes} }

%write Matlab code
\usepackage{listings}
 \definecolor{dkgreen}{rgb}{0,0.6,0}
\definecolor{gray}{rgb}{0.5,0.5,0.5}
\definecolor{mauve}{rgb}{0.58,0,0.82}
\lstset{frame=tb,
  language=Matlab,
  aboveskip=3mm,
  belowskip=3mm,
  showstringspaces=false,
  columns=flexible,
  basicstyle={\small\ttfamily},
  numbers=none,
  numberstyle=\tiny\color{gray},
  keywordstyle=\color{blue},
  commentstyle=\color{dkgreen},
  stringstyle=\color{mauve},
  breaklines=true,
  breakatwhitespace=true
  tabsize=3
}

\usepackage[lastexercise]{exercise}

\newtheorem{ex}{Exercise}
\newtheorem{property}{Property}
\newtheorem{ag}{Algorithm}
\newtheorem{remark}{Remark}
\newtheorem{den}{definition}
\newtheorem{assumption}{Assumption}


\usepackage[nosolutionfiles]{answers}
\Newassociation{sol}{Solution}{ans}



\usepackage{empheq}
\usepackage{comment}
%\usepackage{lscape}
\usepackage{multirow}
\usepackage{url,hyperref}

\hypersetup{
 %   bookmarks=true,         % show bookmarks bar?
    unicode=false,          % non-Latin characters in Acrobat's bookmarks
    pdftoolbar=true,        % show Acrobat's toolbar?
    pdfmenubar=true,        % show Acrobat's menu?
    pdffitwindow=false,     % window fit to page when opened
    pdfstartview={FitH},    % fits the width of the page to the window
    pdftitle={My title},    % title
    pdfauthor={Author},     % author
    pdfsubject={Subject},   % subject of the document
    pdfcreator={Creator},   % creator of the document
    pdfproducer={Producer}, % producer of the document
    pdfkeywords={keyword1} {key2} {key3}, % list of keywords
    pdfnewwindow=true,      % links in new window
    colorlinks=true,       % false: boxed links; true: colored links
    linkcolor=red,          % color of internal links (change box color with linkbordercolor)
    citecolor=green,        % color of links to bibliography
    filecolor=magenta,      % color of file links
    urlcolor=cyan           % color of external links
}


\usepackage{subfigure,epsfig,graphicx,graphics}

\DeclareGraphicsRule{.tif}{png}{.png}{`convert #1 `dirname #1`/`basename #1 .tif`.png}
   \DeclareGraphicsExtensions{.pdf}




\newcommand{\hw}{ {\underline{\tt Homework }} }
\newcommand{\hws}{ {\underline{\tt Homework$\star$}} }
\newcommand{\optional}{ {\it optional} }

\newcommand{\MATLAB}{ \texttt{MATLAB}}
\newcommand{\python}{ \texttt{python}}
\newcommand{\Rlang}{ \texttt{R}}
\newcommand{\SAS}{ \texttt{SAS}}
\newcommand{\MC}{Markov Chain}


\newcommand{\tm}{transition matrix}
\newcommand{\rv}{random variable}
\newcommand{\spl} {supervised learning }
 

\newcommand{\dis}{\underline{\tt discussion}: }
\newcommand{\pri}{\underline{\tt principle}: }




\newcommand{\bq}{\scalebox{6}{\textbf{?} }}
\newcommand{\sq}{\scalebox{2}{\textbf{?} }}
\newcommand{\ck} {  {\scalebox{0.8} {\Interval}   } }

\newcommand{\eps}{\varepsilon}
\newcommand{\To}{\longrightarrow}

% 
\newcommand{\Dcal}{\mathtt{D}}
\newcommand{\Hcal}{\mathcal{H}}
\newcommand{\Ecal}{\mathcal{E}}
\newcommand{\Xcal}{\mathcal{X}}
\newcommand{\Ycal}{\mathcal{Y}}
\newcommand{\Zcal}{\mathcal{Z}}

%%Calculus 

\renewcommand{\d}{\ensuremath{\mathrm{d}}}
\newcommand{\dt}{ \ensuremath{\mathrm{d} t } }
\newcommand{\dx}{ \ensuremath{\mathrm{d} x} }
\newcommand{\dy}{ \ensuremath{\mathrm{d} y } }

%indicator function
\newcommand{\indf}{ \ensuremath{\mathbf{1} } }



%probability
\newcommand{\p}{ \mathbb{P}}
\newcommand{\prob}{{\Pr}}
\newcommand{\PP}{\mbox{PP}}%Poisson process
%condition prob
\newcommand{\cPr}[2]{{\Pr\left(#1\mid #2\right)}}

\newcommand{\FF}{{\mathbb{F}}}

\newcommand{\e}{ \operatorname{\mathbb E}}
\newcommand{\Var}{\operatorname{\mathbb{V} }}
\newcommand{\var}{\operatorname{\text{Var} }}
\newcommand{\MSE}{\operatorname{\text{MSE} }}

\newcommand{\Std}{\operatorname{std}}
\newcommand{\Cov}{\operatorname{cov}}

%Matrix  %mathbf
\newcommand{\Pb}{{\mathbf{P}}}
\newcommand{\Qb}{{\mathbf{Q}}}
\newcommand{\Mb}{{\mathbf{M}}}
\newcommand{\cb}{\mathbf{c}}
\newcommand{\bb}{{\mathbf{b}}}

\newcommand{\Tb}{\mathbf{T}}

\newcommand{\Wb}{\mathbf{W}}
\newcommand{\wb}{\mathbf{w}}
\newcommand{\Xb}{\mathbf{X}}

\newcommand{\xb}{\mathbf{x}}

\newcommand{\Wtn}{\mathbb{W}}
\newcommand{\btn}{\mathbf{b}}



\newcommand{\eye}{{\mathbf{I}}}
%identity matrix
\newcommand{\onem}{{\mathbb{1}}}
\newcommand{\idor}{\mathbf{1}}
\newcommand{\ii}{\mathbf{i}}
%imaginary symbol

\usepackage{tikz}

%State number
\newcommand{\snum}[1]{ \raisebox{.5pt}{\textcircled{\raisebox{-.9pt} {#1}}}}

 \usetikzlibrary{arrows}
\usetikzlibrary{shapes}

%\newcommand{\snum}[1]{%
 % \tikz[baseline=(char.base)]\node[anchor=south west, draw,rectangle, rounded corners, inner sep=1.4pt, minimum size=5mm,
   % text height=1.3mm](char){\ensuremath{#1}} ;}

\newcommand*\circled[1]{\tikz[baseline=(char.base)]{
            \node[shape=circle,draw,inner sep=.4pt] (char) {#1};}}


%real number
\newcommand{\Real}{{\mathbb{R}}}
%integer
\newcommand{\ZZ}{\mathbb{Z}}
%positive integer
\newcommand{\NN}{\mathbb{N}}



\newcommand{\inpd}[2]{\left\langle #1, #2 \right\rangle}
\newcommand{\abs}[1]{\left\vert#1\right\vert}
\newcommand{\norm}[1]{\left\|#1\right\|}
\newcommand{\wt}[1]{{\widetilde{#1}}}
\newcommand{\set}[1]{\left\{#1\right\}}
\newcommand{\partiald}[2]{  \frac{\partial #1 }{\partial #2}}



\newcommand{\ie}{{\it{i.e.}}}



\newcommand{\transpose}{\textsf{T}} % or, \intercal
\newcommand{\diag}{\textsf{diag}}
\newcommand{\tr}{{\textsf{T}}}
\newcommand{\rt}{{\textbf{r}}}

\DeclareMathOperator{\trace}{Trace}


\newcommand{\argmin}{ \operatornamewithlimits{argmin} }
\newcommand{\argmax}{ \operatornamewithlimits{argmax} }




\def\biz{\begin{itemize} }
\def\bizp{\begin{itemize}[<+->] }
\def\eiz{\end{itemize}}


\def\bfm{\begin{frame}}
\def\efm{\end{frame}}

\def\bena{\begin{enumerate}[<+-| alert@+>]}
\def\ben{\begin{enumerate}}
\def\een{\end{enumerate}}


\def\bbk{\begin{block} }
\def\ebk{\end{block}}






\makeatletter
%%%%%%%%%%%%%%%%%%%%%%%%%%%%%% Textclass specific LaTeX commands.
 % this default might be overridden by plain title style

%%%%%%%%%%%%%%%%%%%%%%%%%%%%%% User specified LaTeX commands.
%\usetheme{Warsaw}
\usetheme{Boadilla}
% or ...



%\setbeamertemplate{footline}[text line]{} % makes the footer EMPTY
%\setbeamertemplate{footline}[page number]{} % makes the footer EMPTY

%\usecolortheme{orchid} %not use is better 

\setbeamertemplate{footline}[text line]{%
  \parbox{\linewidth}{\vspace*{-2pt}Xiang Zhou\hfill CityU\hfill \insertpagenumber}}
%\setbeamertemplate{navigation symbols}{}

%\setbeamercovered{transparent}
% or whatever (possibly just delete it)


%\usepackage{babel}
\makeatother



 %
%\addtobeamertemplate{frametitle}{}{%
%\begin{tikzpicture}[remember picture,overlay]
%\node[anchor=south east,yshift=2pt] at (current page.south east) {\includegraphics[height=0.6cm]{CityU_Logo_Basic_Signature.eps}};
%\end{tikzpicture}}
%

\beamerdefaultoverlayspecification{<+->}
%the presentation acts as though a \pause command has been inserted between every two bullets, without the actual need to write \pause after each item.



\begin{document}





\title{Introduction to Stochastic Process}


\author{Xiang Zhou}


\date[CFP 2003]{2013-2014, Semester A}

\maketitle 


%\pgfdeclareimage[height=0.5cm]{institution-logo}{institution-logo-filename}

%\logo{\pgfuseimage{institution-logo}}



%\beamerdefaultoverlayspecification{<+->}

\frame{
\
{\Large Chapter 4: Continuous-Time Markov Chain}


\bigskip
\biz
\item occupancy time
\item limiting behaviour
\item first-passage time

\eiz
 }



\setlength{\belowdisplayskip}{3pt} \setlength{\belowdisplayshortskip}{3pt}
\setlength{\abovedisplayskip}{3pt} \setlength{\abovedisplayshortskip}{3pt}

 


\frame{{Occupancy Times}
Let $\Pm(t)=[P_{ij}(t)]$ be the transition matrix of a CTMC
$\{ X(t): t\in \Real_+\}$.  
The occupancy time of state $j$ is defined as 
\[m_{ij}(T) = \e\left(\left.\int_0^T 1_{\{X(t)=j\}} \ \mathrm{d}t\right|X(0)=i\right),\]
 which is the expected amount of time the CTMC
spends in state $j$ during the interval $[0,T]$.
We compute
\begin{equation*}
\begin{split}
m_{ij}(T)&=\int_0^T\e\left(\left. 1_{\{X(t)=j\}} \right|X(0)=i\right)\mathrm{d}t
=\int_0^T\p\left(\left.X(t)=j \right|X(0)=i\right)\ \mathrm{d}t\\
&=\int_0^T P_{ij}(t)\ \mathrm{d}t.
\end{split}
\end{equation*}

\pause
\begin{theorem}[Thm 4.4]
Define {  \bf occupancy matrix   } $M(T):=[m_{ij}(T)]$, then 
\[M(T)=\int_0^T \Pm(t)\ \mathrm{d}t  \]
\end{theorem}

}



%
%\frame{{Uniformization Method of Computing Occupancy Times\footnote{optional, not required}}
%
%Recall that Thm 4.4 asserts
%\[
%P(t)=\sum_{k=0}^\infty\mathrm{e}^{-rt}\frac{(rt)^k}{k!}\hat{P}^k.
%\]
%So by Thm 4.4,
%\begin{equation*}
%%\begin{split}
%M(T)=\int_0^T P(t)\ \mathrm{d}t=\int_0^T \sum_{k=0}^\infty\mathrm{e}^{-rt}\frac{(rt)^k}{k!}\hat{P}^k\ \mathrm{d}t
%=\sum_{k=0}^\infty\hat{P}^k\int_0^T \mathrm{e}^{-rt}\frac{(rt)^k}{k!}\ \mathrm{d}t.
%%\end{split}
%\end{equation*}
%Using integration by parts repeatedly, we get
%\[
%\int_0^T \mathrm{e}^{-rt}\frac{(rt)^k}{k!}\ \mathrm{d}t=\frac{1}{r}\biggl(1-\sum_{\ell=0}^k\mathrm{e}^{-rT}\frac{(rT)^\ell}{\ell!}\biggr).
%\]
%
%\begin{theorem}[Thm 4.5]
%\[M(T)=\frac{1}{r}\sum_{k=0}^\infty\p(Y>k)\hat{P}^k, \]
%where  $Y\sim \mathrm{Poi}(rT)$.
%\end{theorem}
%}

\frame{{Example: Two-State CTMC}
\begin{example}
Consider a two-state CTMC with the infinitesimal generator
\[ Q=\begin{bmatrix}
-\alpha & \alpha \\
\beta & -\beta
\end{bmatrix},\]
with $\alpha,\beta\geq 0$. Find $M(T)$.
\end{example}
\pause
\vspace{15pt}
We already obtained
\begin{equation*}
P(t)=\frac{1}{\alpha+\beta}\begin{bmatrix}
\beta & \alpha \\
\beta & \alpha
\end{bmatrix}
+\frac{1}{\alpha+\beta}
\begin{bmatrix}
\alpha & -\alpha \\
-\beta & \beta
\end{bmatrix}
\mathrm{e}^{-(\alpha+\beta)t}.
\end{equation*}
Then using Thm 4.4, we get
\begin{equation*}
\begin{split}
M(T)&=\int_0^T \Pm(t)\ \mathrm{d}t\\&=\frac{T}{\alpha+\beta}\begin{bmatrix}
\beta & \alpha \\
\beta & \alpha
\end{bmatrix}
+\frac{1}{(\alpha+\beta)^2}
\begin{bmatrix}
\alpha & -\alpha \\
-\beta & \beta
\end{bmatrix}
\left(1-\mathrm{e}^{-(\alpha+\beta)T}\right).
\end{split}
\end{equation*}

}


\frame{ Limiting behavior}


\frame{ {Limiting behavior}

\begin{definition}
The limiting distribution of   CTMC  $\{ X(t): t\in \Real_+\}$ 
with transition semigroup $\{\Pm(t)=[P_{ij}(t)]\}$ is
$\pi=[\pi_j]$ where
\[ \pi_j:=\lim_{t\to\infty} P_{ij}(t) \]
provided the above limit exists and is independent $i$, i.e.,
the limit $\lim_{t\to\infty}\Pm(t)$ have the same row vectors, which is $\pi$.
\end{definition}
\pause 

\begin{definition}
  CTMC  $\{ X(t): t\in \Real_+\}$ is {\bf  irreducible}
if its embedded DTMC
$\{Z_n:n=0,1,\ldots\}$ is irreducible.
\end{definition}
}

\frame{{Balance Equation}

\begin{theorem}[Thm 4.6,4.7,4.8]
\biz
\item
An irreducible CTMC  $\{ X(t): t\in \Real_+\}$
with (strictly positive) rate matrix $R=[r_{ij}]$ has a unique
limiting distribution $\pi$.
\item $\pi$ is
the solution to the balance equation and the normalizing equation
\begin{equation*}
\boxed{\pi_j r_j =\sum_{i} \pi_i r_{i,j}, \quad \sum_{i}\pi_i =1.}
\end{equation*}
\item  $\pi$ is also the stationary distribution of   $X$.
\item  $\pi$ is also the occupancy  distribution of   $X$, i.e., $\displaystyle\lim_{T\rightarrow\infty}\frac{m_{ij}(T)}{T}=\pi_j$.
\eiz
\end{theorem}


The balance equation can be rewritten as \[\boxed{\pi Q=0}.\]
By Kolmogorov equation, the distribution $\pi(t):=\pi(0)\Pm(t)$ of $X(t)$ satisfies
\[
\pi'(t)=\pi(0)\Pm'(t)=\pi(0)\Pm(t)Q=\pi(t)Q.
\]
So $\pi(t)\equiv \pi(0)$ if and only if $\pi(0)Q=0$.
}

\frame{{Example(Example 4.22): Two-State CTMC}

Consider a two-state CTMC with the infinitesimal generator
\[ Q=\begin{bmatrix}
-\alpha & \alpha \\
\beta & -\beta
\end{bmatrix},\quad \alpha,\beta\geq 0.\]
It is irreducible. Hence it has a unique limiting distribution $[\pi_0, \pi_1]$.
Solving simultaneously the two balance equations
\begin{equation*}
\begin{split}
\alpha \pi_0&=\beta \pi_1,\\
\beta \pi_1&=\alpha \pi_0
\end{split}
\end{equation*}
and the normalizing equation
\[\pi_0+\pi_1=1\]
yields
\[\pi_0=\frac{\beta}{\alpha+\beta}, \quad \pi_1=\frac{\alpha}{\alpha+\beta}.\]
Note that this is also the stationary distribution and the occupancy distribution of the CTMC.
}


\frame{{Example: Birth and Death Process}
Consider the birth and death process on $\{0,1,...,N\}$ with the generator
\[ Q=\begin{bmatrix}
-\lambda_0 & \lambda_0 & 0 & 0 & \cdots &\cdots &\cdots & 0 & 0 \\
\mu_1 & -\lambda_1-\mu_1 & \lambda_1 & 0 & \cdots &\cdots &\cdots & 0 & 0 \\
0 & \mu_2 &-\lambda_2-\mu_2&  \lambda_2  & \cdots&\cdots &\cdots & 0 & 0 \\
\vdots & \vdots & \vdots & \vdots & \ddots & \ddots & \ddots &\vdots&\vdots\\
0 & 0 & 0 & \vdots & \ddots & \ddots & \mu_{N-1} &-\lambda_{N-1}-\mu_{N-1}&\lambda_{N-1}\\
0 & 0 & 0 & 0 & \ddots & \ddots & \ddots &\mu_N&-\mu_N
\end{bmatrix}.\]
The balance equation reads
\begin{equation*}
\begin{cases}
&0=-\lambda_0\pi_0+\mu_1\pi_1,\\
&0=\lambda_0\pi_0-(\lambda_1+\mu_1)\pi_1+\mu_2\pi_2,\\
&\vdots\\
&0=\lambda_{j-1}\pi_{j-1}-(\lambda_j+\mu_j)\pi_j+\mu_{j+1}\pi_{j+1},\\
&\vdots\\
&0=\lambda_{N-1}\pi_{N-1}-\mu_N\pi_N.
\end{cases}
\end{equation*}
}


\renewcommand{\thefootnote}{\fnsymbol{footnote}}
\setcounter{footnote}{1}

\frame{{Example: Birth and Death Process}
\small
$0=-\lambda_0\pi_0+\mu_1\pi_1$ gives $\pi_1=\frac{\lambda_0}{\mu_1}\pi_0$,\\
$0=\lambda_0\pi_0-(\lambda_1+\mu_1)\pi_1+\mu_2\pi_2$ gives $\pi_2=-\frac{\lambda_0}{\mu_2}\pi_0+\frac{\lambda_1+\mu_1}{\mu_2}\pi_1
=-\frac{\lambda_0}{\mu_2}\pi_0+\frac{\lambda_1+\mu_1}{\mu_2}\frac{\lambda_0}{\mu_1}\pi_0
=\frac{\lambda_1\lambda_0}{\mu_2\mu_1}\pi_0$,\\
$\vdots$\\
$0=\lambda_{j-1}\pi_{j-1}-(\lambda_j+\mu_j)\pi_j+\mu_{j+1}\pi_{j+1}$ gives
$\pi_{j+1}=-\frac{\lambda_{j-1}}{\mu_{j+1}}\pi_{j-1}+\frac{\lambda_j+\mu_j}{\mu_{j+1}}\pi_j
=-\frac{\lambda_{j-1}}{\mu_{j+1}}\frac{\lambda_{j-2}\cdots\lambda_0}{\mu_{j-1}\cdots\mu_1}\pi_{0}
+\frac{\lambda_j+\mu_j}{\mu_{j+1}}\frac{\lambda_{j-1}\cdots\lambda_0}{\mu_j\cdots\mu_1}\pi_0
=\frac{\lambda_j\cdots\lambda_0}{\mu_{j+1}\cdots\mu_1}\pi_0$,\\
$\vdots$\\
$0=\lambda_{N-1}\pi_{N-1}-\mu_N\pi_N$ gives
$\pi_{N}=\frac{\lambda_{N-1}}{\mu_N}\pi_{N-1}
=\frac{\lambda_{N-1}}{\mu_N}\frac{\lambda_{N-2}\cdots\lambda_0}{\mu_{N-1}\cdots\mu_1}\pi_0
=\frac{\lambda_{N-1}\cdots\lambda_0}{\mu_{N}\cdots\mu_1}\pi_0$.

Then  the normalizing equation
\[
1=\sum_{i=0}^N\pi_i=\pi_0\sum_{i=0}^N\frac{\lambda_{i-1}\cdots\lambda_0}{\mu_{i}\cdots\mu_1}
\footnote{The conventions $\lambda_{i-1}\cdots\lambda_0=\prod_{j=0}^{i-1}\lambda_j=0$ for $i=0$ and
$\mu_{i}\cdots\mu_{1}=\prod_{j=1}^i\mu_j=0$ for $i=0$ are adopted.}\Longrightarrow
\pi_0=\frac{1}{\sum_{i=0}^N\frac{\lambda_{i-1}\cdots\lambda_0}{\mu_{i}\cdots\mu_1}},
\]
and so
\[
\pi_j=\frac{\lambda_{j-1}\cdots\lambda_0}{\mu_j\cdots\mu_1\sum_{i=0}^N\frac{\lambda_{i-1}\cdots\lambda_0}{\mu_{i}\cdots\mu_1}},
\quad  j=0, \cdots, N.
\]
}






\frame{{Expected Total Cost}

Assume that whenever the CTMC is in state $i$, it incurs costs at rate $c(i)$.

%\vspace{10pt}

\biz
\item   The cost rate at time $t$ is $c\bigl(X(t)\bigr)$.
\item   The total cost up to time $T$ is given by $\int_0^T c\bigl(X(t)\bigr) \mathrm{d}t$.
\item   The expected total cost up to $T$, starting from state $i$, is given by
\[g(i,T)=\e\left(\left.\int_0^T c\bigl(X(t)\bigr) \ \mathrm{d}t\right|X(0)=i\right).\]
\eiz
\begin{equation*}
\begin{split}
g(i,T)&=\int_0^T\e\left(\left. c\bigl(X(t)\bigr) \right|X(0)=i\right)\mathrm{d}t\\
&=\int_0^T\sum_{j=1}^Nc(j)\p\left(\left.X(t)=j \right|X(0)=i\right)\mathrm{d}t\\
&=\sum_{j=1}^Nc(j)\int_0^T P_{ij}(t)\ \mathrm{d}t=\sum_{j=1}^Nc(j)m_{i,j}(T).
\end{split}
\end{equation*}
}

\frame{
\begin{theorem}[Thm 4.9]
Write   $c=\begin{bmatrix}c(1)\\\vdots\\c(N)\end{bmatrix}$ and   $g(T)=\begin{bmatrix}g(1,T)\\\vdots\\g(N,T)\end{bmatrix}$ as column vectors,  then the expected  total cost is 
\[g(T)=M(T)c\]
where $M(T)$ is the occupancy matrix.
\end{theorem}
}


\frame{{Example(Example 4.28): Two-State CTMC}
\begin{example}
Consider a two-state CTMC with the infinitesimal generator
\[ Q=\begin{bmatrix}
-\alpha & \alpha \\
\beta & -\beta
\end{bmatrix},\quad \alpha,\beta\geq 0.\]
Set $c(0)=-B$, $c(1)=A$ where $A$ and $B$ are two positive numbers.
 Find the expected total cost $g(T)$.
\end{example}
\pause
\vspace{15pt}

We already computed
\[
M(T)=\frac{T}{\alpha+\beta}\begin{bmatrix}
\beta & \alpha \\
\beta & \alpha
\end{bmatrix}
+\frac{1}{(\alpha+\beta)^2}
\begin{bmatrix}
\alpha & -\alpha \\
-\beta & \beta
\end{bmatrix}
\left(1-\mathrm{e}^{-(\alpha+\beta)T}\right).
\]
Then by Thm 4.9, we have
\begin{equation*}
\begin{split}
g(T)&=M(T)c=\left(\frac{T}{\alpha+\beta}\begin{bmatrix}
\beta & \alpha \\
\beta & \alpha
\end{bmatrix}
+\frac{1-\mathrm{e}^{-(\alpha+\beta)T}}{(\alpha+\beta)^2}
\begin{bmatrix}
\alpha & -\alpha \\
-\beta & \beta
\end{bmatrix}\right)
\begin{bmatrix}-B\\A\end{bmatrix}\\
&=\frac{\left(\alpha A-\beta B\right)T}{\alpha+\beta}\begin{bmatrix}1\\1\end{bmatrix}+\frac{B+A}{(\alpha+\beta)^2}\left(1-\mathrm{e}^{-(\alpha+\beta)T}\right)
\begin{bmatrix}-\alpha\\\beta\end{bmatrix}.
\end{split}
\end{equation*}

}



\frame{{Long-run Cost Rates}

\biz
\item   The long-run cost rate is defined as
\[g(i)=\lim_{T\rightarrow\infty}\frac{g(i,T)}{T}.\]
\eiz

\begin{theorem}[Thm 4.10]
Suppose $\{ X(t): t\in \Real_+\}$ is an irreducible CTMC with limiting distribution $\pi=[\pi_1, \cdots, \pi_N]$. Then
\[g=g(i)=\sum_{j=1}^N\pi_jc(j), \quad 1\le i\le N. \]
\end{theorem}
\begin{equation*}
\begin{split}
g(i)&=\lim_{T\rightarrow\infty}\frac{g(i,T)}{T}=\lim_{T\rightarrow\infty}\frac{\sum_{j=1}^Nm_{i,j}(T)c(j)}{T}\\
&=\sum_{j=1}^Nc(j)\lim_{T\rightarrow\infty}\frac{m_{i,j}(T)}{T}=\sum_{j=1}^Nc(j)\pi_j.
\end{split}
\end{equation*}
%Intuition: the CTMC incurs a cost at rate $c(j)$

}

\frame{{Example(Example 4.30): Two-State CTMC}
\begin{example}
Consider a two-state CTMC with the infinitesimal generator
\[ Q=\begin{bmatrix}
-\alpha & \alpha \\
\beta & -\beta
\end{bmatrix},\quad \alpha,\beta\geq 0.\]
Set $c(0)=-B$, $c(1)=A$. Find the long-run cost rate $g$.
\end{example}
\pause
\vspace{15pt}

We already computed
\[\pi_0=\frac{\beta}{\alpha+\beta}, \quad \pi_1=\frac{\alpha}{\alpha+\beta}.\]
Then by Thm 4.10, we get
\[g=-B\pi_0+A\pi_1=\frac{A\alpha-B\beta}{\alpha+\beta}.\]

}



\frame{{First-Passage Times}
\begin{definition}
The expected first-passage time to a subset $A\subset S$ is 
defined as  $h_A(i)=\e(T_A|X(0)=i)$, where $T_A=\inf \{t\geq 0: X(t) \in A \}$ is the first-passage time into a set $A$.
\end{definition}
\vspace{10pt}
\pause

Start from time $t=0$ now. 
Let  $\tau$  be on the time of the first jump of $X$.
Conditioning on  the state $X(\tau)$ of the CTMC
 (i.e., the embedded chain $Z_1$), 
we perform the following ``one-more-step analysis",
\begin{equation*}
\begin{split}
h_A(i)&=\e(T_A|X(0)=i)=\sum_{j}\e(T_A|X(0)=i,X(\tau)=j)\p(X(\tau)=j|X(0)=i)\\
&=\sum_{j}\bigl[\e(\tau|X(0)=i)+\e(T_A|X(0)=\hb{j})\bigr]\p(X(\tau)=j|X(0)=i)\\
&=\sum_{j}\left(\frac{1}{r_i}+h_A(j)\right)\frac{r_{i,j}}{r_i}
=\frac{1}{r_i^2}\sum_{j}r_{i,j}+\frac{1}{r_i}\sum_{j}r_{i,j}h_A(j)=\frac{1}{r_i}+\frac{1}{r_i}\sum_{j\notin A}r_{i,j}h_A(j).
\end{split}
\end{equation*}
{\small 
Here  we use the facts that
the expectation of the first jump  time
$\e(\tau|X(0)=i)=\frac{1}{r_i}$
and transition probability of the embedded DTMC $p_{i,j}=\frac{r_{i,j}}{r_i}$,
as well as $r_i=\sum_j r_{ij}$.
}

}
\frame{{First-passage time Formula}
\begin{theorem}[Thm 4.11]
The mean first passage times $\{h_A(i):i\notin A\}$ satisfy
\[r_ih_A(i)=1+\sum_{j\notin A}r_{i,j}h_A(j), \quad i\notin A,\]
and the boundary condition $h_A(i)=0$ for $i\in A$.
\end{theorem}

}

\frame{
\begin{example}[Example 4.32]
Consider a CTMC with the infinitesimal generator
\[ Q=\begin{bmatrix}
-5& 2 & 3 & 0 \\
4 & -6 & 2 & 0\\
0 & 2 & -4 & 2\\
1 & 0 & 3 & -4
\end{bmatrix}.\]
Compute the expected time to reach state $4$.
\end{example}
\pause
\vspace{15pt}

By Thm 4.11, we have
\begin{equation*}
\begin{cases}
5h_{\{4\}}(1)=1+2h_{\{4\}}(2)+3h_{\{4\}}(3),\\
6h_{\{4\}}(2)=1+4h_{\{4\}}(1)+2h_{\{4\}}(3),\\
4h_{\{4\}}(1)=1+2h_{\{4\}}(2).
\end{cases}
\end{equation*}
Solving simultaneously, we get
\[
h_{\{4\}}(1)=\frac{14}{11}\approx1.2727, \quad h_{\{4\}}(2)=\frac{29}{22}\approx1.3182, \quad h_{\{4\}}(3)\approx\frac{10}{11}=0.9091.
\]
}

\frame{
\begin{example}[ Birth and Death Process]
Consider the birth and death process on $\{0,1,...,N\}$ with the generator
\[ Q=\begin{bmatrix}
-\lambda & \lambda & 0 & 0 & \cdots &\cdots &\cdots & 0 & 0 \\
\mu & -\lambda-\mu & \lambda & 0 & \cdots &\cdots &\cdots & 0 & 0 \\
%0 & \mu &-\lambda-\mu&  \lambda  & \cdots&\cdots &\cdots & 0 & 0 \\
\vdots & \vdots & \vdots & \vdots & \ddots & \ddots & \ddots &\vdots&\vdots\\
0 & 0 & 0 & \vdots & \ddots & \ddots & \mu &-\lambda-\mu&\lambda\\
0 & 0 & 0 & 0 & \ddots & \ddots & \ddots &\mu &-\mu
\end{bmatrix}.\]
Compute the expected time to reach states $0$ and $N$.
\end{example}
\pause
\vspace{10pt}

By Thm 4.11, we have the boundary conditions $h_{\{0,N\}}(0)=h_{\{0,N\}}(N)=0$ and
\[
h_{\{0,N\}}(i)=\frac{1}{\lambda+\mu}+\frac{\mu}{\lambda+\mu}h_{\{0,N\}}(i-1)+\frac{\lambda}{\lambda+\mu}h_{\{0,N\}}(i+1),
\quad 1\le i\le N-1.
\]
We can verify that the solution is given by
\[
h_{\{0,N\}}(i)=
\begin{cases}
\frac{1}{\mu-\lambda}\Bigl(i-N\frac{1-(\mu/\lambda)^i}{1-(\mu/\lambda)^N}\Bigr),\quad &\text{if} \quad \mu\neq \lambda,\\
\frac{1}{2\mu}i(N-i), &\text{if} \quad \mu=\lambda,
\end{cases}
\quad  i=0,\cdots, N.
\]


}

\frame{{\hw}
\biz 
 \item
TEXTBOOK:  
 Page 142: 4.11,  4.20,   4.22, 4.32
\item 
 %The book of  "Understanding Markov Chain" Ex.  10.11. (see below)
%  (solution P327, under Ex.10.1)
A workshop has five machines and one repairman. Each machine functions until it fails at an exponentially distributed random time with parameter $0.20$ per hour. On the other hand, it takes a exponentially distributed random time with parameter (rate) $0.50$ per hour to repair a given machine. We assume that the machines behave independently of one another, and that
(a) up to five machines can operate at any given time, (b) at most one can be under repair at any time.
Compute the proportion of time the repairman is idle in the long run.
\eiz
}

\frame{{\hw}
\biz 
 \item 
 %The book of  "Understanding Markov Chain" Ex.  10.11. (see below)
%  (solution P327, under Ex.10.7)

A system consists of two machines and two repairmen. Each machine can work until failure at an exponentially distributed random time with parameter $0.2$. A failed machine can be repaired only be a single repairman, within an exponentially distributed random time with parameter $0.25$. We model the number$  X(t)$ of working machines at time $t\in\Real_+$ as a continuous-time Markov process.
 \ben
 \item Complete the missing entries in the generator  matrix
\[ Q=\begin{bmatrix}
\square  &0.5 & \square \\
0.2  & \square & \square \\
0 & \square & -0.4
\end{bmatrix}.\] 

\item 
Calculate the long-run probability distribution $(\pi_0,\pi_1,\pi_2)$ of $X(t)$.

\item Compute the average number of working machines in the long
run.
\item Given that a working machine can produce  $100$ units every hour, how many units can the system produce per hour in the long run?
\een
 
\eiz 
}
  




\end{document}
