\documentclass[a4 paper]{article}
% Set target color model to RGB
\usepackage[inner=2.0cm,outer=2.0cm,top=2.5cm,bottom=2.5cm]{geometry}
\usepackage{setspace}
\usepackage[rgb]{xcolor}
\usepackage{verbatim}
\usepackage{subcaption}
\usepackage{amsgen,amsmath,amstext,amsbsy,amsopn,tikz,amssymb}
\usepackage{fancyhdr}
\usepackage[colorlinks=true, urlcolor=blue,  linkcolor=blue, citecolor=blue]{hyperref}
\usepackage[colorinlistoftodos]{todonotes}
\usepackage{rotating}
%\usetikzlibrary{through,backgrounds}
\hypersetup{%
	pdfauthor={Ashudeep Singh},%
	pdftitle={Homework},%
	pdfkeywords={Tikz,latex,bootstrap,uncertaintes},%
	pdfcreator={PDFLaTeX},%
	pdfproducer={PDFLaTeX},%
}
%\usetikzlibrary{shadows}
% \usepackage[francais]{babel}
\usepackage{booktabs}
%\input{macros.tex}

\newcommand{\ra}[1]{\renewcommand{\arraystretch}{#1}}

\newtheorem{thm}{Theorem}[section]
\newtheorem{prop}[thm]{Proposition}
\newtheorem{lem}[thm]{Lemma}
\newtheorem{cor}[thm]{Corollary}
\newtheorem{defn}[thm]{Definition}
\newtheorem{rem}[thm]{Remark}
\numberwithin{equation}{section}

\newcommand{\homework}[6]{
	\pagestyle{myheadings}
	\thispagestyle{plain}
	\newpage
	\setcounter{page}{1}
	\noindent
	\begin{center}
		\framebox{
			\vbox{\vspace{2mm}
				\hbox to 6.28in { {\bf MA 6630:~ Introduction to Statistical Learning \hfill {\small #2}} }
				\vspace{6mm}
				\hbox to 6.28in { {\Large \hfill #1  \hfill} }
				\vspace{6mm}
				\hbox to 6.28in { {\it Instructor: {\rm #3} \hfill Teaching Assistant: {\rm #5}} }
				%\hbox to 6.28in { {\it TA: #4  \hfill #6}}
				\vspace{2mm}}
		}
	\end{center}
	\markboth{#5 -- #1}{#5 -- #1}
	\vspace*{4mm}
}

\newcommand{\problem}[2]{~\\\fbox{\textbf{Problem #1}}\hfill (#2 points)\newline\newline}
\newcommand{\subproblem}[1]{~\newline\textbf{(#1)}}
\newcommand{\0}{\mathbf{0}}


\begin{document}
	\homework{Assignment 5}{2022-2023 Semester B}{Dr. ZHOU Xiang}{}{WANG Yue}
	
	
	
 
	\problem{1}{5}
	Suppose the dataset for the binary classification has $n=4$ samples in $\mathbb{R}^2$ plane as follows:
	
	\begin{tabular}{ll ll}
		$ {\bf x}_1$=(1, -1),&  $y_1$ = -1;
		${\bf x}_2$=(1, 1), &  $y_2$ =   1.\\
		${\bf x}_3$=(-1, 1),&  $y_3$ = -1;
		${\bf x}_4$=(-1, -1),&  $y_4$ = 1.
	\end{tabular}.

	Basically, $y=x_1*x_2$ (Boolean logic operation XOR).
	Is this dataset linearly separable? 
	Can you find the maximal margin classifier 
	$f({\bf x})={\bf w}\cdot {\bf x}+ b$?
	What 
	is the decision boundary of this classifier?
	What is the margin of this dataset? 
	Use the computer to solve the support vector classifier 
	by using the   penalty $C\sum_{i}\xi_i$
	and show how the result changes with increasing $C$.
	%\end{ex}
	
	\bigskip
	\noindent\textbf{Solution:}	   
	
	
	\begin{align*}
		&\max_{{\bf w}\in \mathbb{R}^d,b\in \mathbb{R}} M
		\\
		\mbox{subject to} ~&~~
		w_1^2 + w_2^2 =1\\
		&  w_1 + w_2 + b \geq M  
		\\
		&  -w_1 - w_2 + b \geq M  
		\\&
		w_1 - w_2 - b \geq M  
		\\&
		-w_1 + w_2 - b \geq M  
	\end{align*} 
	
	The four linear constraints are equivalent to 
	$0
	\le |{w_1+w_2}|\leq -M+b$ and $0\le |{w_1-w_2}|\leq -M-b$.
	
	Then $|{w_1}|\le \frac12 |w_1+w_2|+\frac12 |w_1-w_2|\leq -M$. So any admissible $M$ is negative
	and thus the dataset is {\bf not linearly separable} (you can also see this by plotting four points in the plane).
	
	We also have 
	$-M\pm b\geq 0$. So the possible max of $M$ is $M^*=b$ or $M^*=-b$.
	
	If $M^*=b$, then $w_1+w_2=0\Rightarrow w_1=\pm 1/\sqrt{2}$,
	and $|w_1|=|w_2|\le -M\Rightarrow M\leq -|w_1|$,
	So $M^*=b^*=-|w_1|$ and 
	$f(x)=(-x_1+x_2-1)/\sqrt{2}$ 
	or $f(x)=( x_1-x_2-1)/\sqrt{2}$.
	
	Likewise, if $M^*=-b$, then 
	$f(x)=(-x_1+x_2+1)/\sqrt{2}$
	or $f(x)=( x_1-x_2+1)/\sqrt{2}$.
	
	
	In both cases, $M^*=-1/\sqrt{2}$.
	The decision boundaries of the four maximal margin classifiers are four diagonal lines.
 
	 	\begin{figure}[t!]
	 	\centering
	 	\includegraphics[height=2.3in]{code.png}
	 	\includegraphics[height=2.5in]{fig.png}
	 	%\includegraphics[height=2.5in]{connectivity2.eps}
	 	%\includegraphics[height=2.5in]{connectivity3.eps}
	 	\caption{The margin versus cost graph. } 
	 \end{figure}
	
	
	
\end{document} 