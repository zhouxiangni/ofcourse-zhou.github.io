%\documentclass[english,10pt]{beamer}
\documentclass[english,handout]{beamer}






 
%\usepackage{mathptmx}
%\renewcommand{\sfdefault}{lmss}
\usepackage[T1]{fontenc}
%\usepackage[latin9]{inputenc}
\usepackage[utf8]{inputenc}

\synctex=-1

\usefonttheme{professionalfonts}

%\setbeamertemplate{navigation symbols}{}
%\setbeamertemplate{caption}[numbered]


\useinnertheme{rectangles}
%http://tex.stackexchange.com/questions/11168/change-bullet-style-formatting-in-beamer

 \AtBeginDocument{
  \addtolength\abovedisplayskip{-0.4\baselineskip}%
  \addtolength\belowdisplayskip{-0.4\baselineskip}%
}%change the space between text lines and the math formula


\usepackage{pifont}
%Postscript ZipfDingbats font
%the command \ding{number}, will print the specified symbol

\usepackage{fontawesome}
%icon package
\DeclareFontFamily{U}{FontAwesomeOne}{}
\DeclareFontShape{U}{FontAwesomeOne}{m}{n}{<-> FontAwesome--fontawesomeone}{}
\DeclareRobustCommand\FAone{\fontencoding{U}\fontfamily{FontAwesomeOne}\fontseries{m}\fontshape{n}\selectfont}
\DeclareFontFamily{U}{FontAwesomeTwo}{}
\DeclareFontShape{U}{FontAwesomeTwo}{m}{n}{<-> FontAwesome--fontawesometwo}{}
\DeclareRobustCommand\FAtwo{\fontencoding{U}\fontfamily{FontAwesomeTwo}\fontseries{m}\fontshape{n}\selectfont}
\DeclareFontFamily{U}{FontAwesomeThree}{}
\DeclareFontShape{U}{FontAwesomeThree}{m}{n}{<-> FontAwesome--fontawesomethree}{}
\DeclareRobustCommand\FAthree{\fontencoding{U}\fontfamily{FontAwesomeThree}\fontseries{m}\fontshape{n}\selectfont}

%ftp://ftp.dante.de/tex-archive/fonts/fontawesome/doc/fontawesome.pdf
%http://tug.ctan.org/info/symbols/comprehensive/symbols-a4.pdf


\usepackage{amsmath,amssymb,amsfonts,bm,mathrsfs,mathtools}

\usepackage{tikzsymbols}
%\usepackage[tikz]{bclogo}



\usepackage{perpage}
\MakePerPage{footnote} %reset for each page
%\renewcommand{\thefootnote}{\fnsymbol{footnote}} %use symbol, limit less than 9 symbols



%%%% HIGHTLIGHT  and annotation &=%%%%%%%%
\usepackage{color,xcolor}
 \usepackage{todonotes}

\usepackage[normalem]{ulem}

\usepackage[many]{tcolorbox}

\tcbset{fonttitle=\scriptsize}
\tcbset{highlight math style={enhanced,
  colframe=red!40!black,colback=yellow!20!white,arc=2pt,boxrule=.2pt,
  }}
  \newtcbox{\otherbox}[1][]{nobeforeafter,math upper,tcbox raise base,
enhanced,frame hidden,boxrule=0pt,interior style={top color=green!10!white,
bottom color=green!10!white,middle color=green!50!yellow},
fuzzy halo=1pt with green,#1}
%%\tcbhighmath{math here}
%% \otherbox{math here}



%%%%% HIGHLIGHT %%%%%%
\newcommand{\hb}[1]{{\color{blue}{#1}}}
%\noindent\rule{\textwidth}{.5pt}

%:
\usepackage{soul}

\newcommand\hcancel[2][black]{\setbox0=\hbox{$#2$}%
\rlap{\raisebox{.45\ht0}{\textcolor{#1}{\rule{\wd0}{1pt}}}}#2}
%cross to delete

\newcommand{\mcb}[2]{\colorbox{#1}{$\displaystyle #2$}}
%highlight math

\newcommand{\hlfancy}[2]{\sethlcolor{#1}\hl{#2}}
%specified color , for\hl

\newcommand\myhl{\bgroup\markoverwith
  {\textcolor{yellow}{\rule[-.5ex]{2pt}{2.5ex}}}\ULon}



\mode<presentation>{ \usetheme{boxes} }

%write Matlab code
\usepackage{listings}
 \definecolor{dkgreen}{rgb}{0,0.6,0}
\definecolor{gray}{rgb}{0.5,0.5,0.5}
\definecolor{mauve}{rgb}{0.58,0,0.82}
\lstset{frame=tb,
  language=Matlab,
  aboveskip=3mm,
  belowskip=3mm,
  showstringspaces=false,
  columns=flexible,
  basicstyle={\small\ttfamily},
  numbers=none,
  numberstyle=\tiny\color{gray},
  keywordstyle=\color{blue},
  commentstyle=\color{dkgreen},
  stringstyle=\color{mauve},
  breaklines=true,
  breakatwhitespace=true
  tabsize=3
}

\usepackage[lastexercise]{exercise}

\newtheorem{ex}{Exercise}
\newtheorem{property}{Property}
\newtheorem{ag}{Algorithm}
\newtheorem{remark}{Remark}
\newtheorem{den}{definition}
\newtheorem{assumption}{Assumption}


\usepackage[nosolutionfiles]{answers}
\Newassociation{sol}{Solution}{ans}



\usepackage{empheq}
\usepackage{comment}
%\usepackage{lscape}
\usepackage{multirow}
\usepackage{url,hyperref}

\hypersetup{
 %   bookmarks=true,         % show bookmarks bar?
    unicode=false,          % non-Latin characters in Acrobat's bookmarks
    pdftoolbar=true,        % show Acrobat's toolbar?
    pdfmenubar=true,        % show Acrobat's menu?
    pdffitwindow=false,     % window fit to page when opened
    pdfstartview={FitH},    % fits the width of the page to the window
    pdftitle={My title},    % title
    pdfauthor={Author},     % author
    pdfsubject={Subject},   % subject of the document
    pdfcreator={Creator},   % creator of the document
    pdfproducer={Producer}, % producer of the document
    pdfkeywords={keyword1} {key2} {key3}, % list of keywords
    pdfnewwindow=true,      % links in new window
    colorlinks=true,       % false: boxed links; true: colored links
    linkcolor=red,          % color of internal links (change box color with linkbordercolor)
    citecolor=green,        % color of links to bibliography
    filecolor=magenta,      % color of file links
    urlcolor=cyan           % color of external links
}


\usepackage{subfigure,epsfig,graphicx,graphics}

\DeclareGraphicsRule{.tif}{png}{.png}{`convert #1 `dirname #1`/`basename #1 .tif`.png}
   \DeclareGraphicsExtensions{.pdf}




\newcommand{\hw}{ {\underline{\tt Homework }} }
\newcommand{\hws}{ {\underline{\tt Homework$\star$}} }
\newcommand{\optional}{ {\it optional} }

\newcommand{\MATLAB}{ \texttt{MATLAB}}
\newcommand{\python}{ \texttt{python}}
\newcommand{\Rlang}{ \texttt{R}}
\newcommand{\SAS}{ \texttt{SAS}}
\newcommand{\MC}{Markov Chain}


\newcommand{\tm}{transition matrix}
\newcommand{\rv}{random variable}
\newcommand{\spl} {supervised learning }
 

\newcommand{\dis}{\underline{\tt discussion}: }
\newcommand{\pri}{\underline{\tt principle}: }




\newcommand{\bq}{\scalebox{6}{\textbf{?} }}
\newcommand{\sq}{\scalebox{2}{\textbf{?} }}
\newcommand{\ck} {  {\scalebox{0.8} {\Interval}   } }

\newcommand{\eps}{\varepsilon}
\newcommand{\To}{\longrightarrow}

% 
\newcommand{\Dcal}{\mathtt{D}}
\newcommand{\Hcal}{\mathcal{H}}
\newcommand{\Ecal}{\mathcal{E}}
\newcommand{\Xcal}{\mathcal{X}}
\newcommand{\Ycal}{\mathcal{Y}}
\newcommand{\Zcal}{\mathcal{Z}}

%%Calculus 

\renewcommand{\d}{\ensuremath{\mathrm{d}}}
\newcommand{\dt}{ \ensuremath{\mathrm{d} t } }
\newcommand{\dx}{ \ensuremath{\mathrm{d} x} }
\newcommand{\dy}{ \ensuremath{\mathrm{d} y } }

%indicator function
\newcommand{\indf}{ \ensuremath{\mathbf{1} } }



%probability
\newcommand{\p}{ \mathbb{P}}
\newcommand{\prob}{{\Pr}}
\newcommand{\PP}{\mbox{PP}}%Poisson process
%condition prob
\newcommand{\cPr}[2]{{\Pr\left(#1\mid #2\right)}}

\newcommand{\FF}{{\mathbb{F}}}

\newcommand{\e}{ \operatorname{\mathbb E}}
\newcommand{\Var}{\operatorname{\mathbb{V} }}
\newcommand{\var}{\operatorname{\text{Var} }}
\newcommand{\MSE}{\operatorname{\text{MSE} }}

\newcommand{\Std}{\operatorname{std}}
\newcommand{\Cov}{\operatorname{cov}}

%Matrix  %mathbf
\newcommand{\Pb}{{\mathbf{P}}}
\newcommand{\Qb}{{\mathbf{Q}}}
\newcommand{\Mb}{{\mathbf{M}}}
\newcommand{\cb}{\mathbf{c}}
\newcommand{\bb}{{\mathbf{b}}}

\newcommand{\Tb}{\mathbf{T}}

\newcommand{\Wb}{\mathbf{W}}
\newcommand{\wb}{\mathbf{w}}
\newcommand{\Xb}{\mathbf{X}}

\newcommand{\xb}{\mathbf{x}}

\newcommand{\Wtn}{\mathbb{W}}
\newcommand{\btn}{\mathbf{b}}



\newcommand{\eye}{{\mathbf{I}}}
%identity matrix
\newcommand{\onem}{{\mathbb{1}}}
\newcommand{\idor}{\mathbf{1}}
\newcommand{\ii}{\mathbf{i}}
%imaginary symbol

\usepackage{tikz}

%State number
\newcommand{\snum}[1]{ \raisebox{.5pt}{\textcircled{\raisebox{-.9pt} {#1}}}}

 \usetikzlibrary{arrows}
\usetikzlibrary{shapes}

%\newcommand{\snum}[1]{%
 % \tikz[baseline=(char.base)]\node[anchor=south west, draw,rectangle, rounded corners, inner sep=1.4pt, minimum size=5mm,
   % text height=1.3mm](char){\ensuremath{#1}} ;}

\newcommand*\circled[1]{\tikz[baseline=(char.base)]{
            \node[shape=circle,draw,inner sep=.4pt] (char) {#1};}}


%real number
\newcommand{\Real}{{\mathbb{R}}}
%integer
\newcommand{\ZZ}{\mathbb{Z}}
%positive integer
\newcommand{\NN}{\mathbb{N}}



\newcommand{\inpd}[2]{\left\langle #1, #2 \right\rangle}
\newcommand{\abs}[1]{\left\vert#1\right\vert}
\newcommand{\norm}[1]{\left\|#1\right\|}
\newcommand{\wt}[1]{{\widetilde{#1}}}
\newcommand{\set}[1]{\left\{#1\right\}}
\newcommand{\partiald}[2]{  \frac{\partial #1 }{\partial #2}}



\newcommand{\ie}{{\it{i.e.}}}



\newcommand{\transpose}{\textsf{T}} % or, \intercal
\newcommand{\diag}{\textsf{diag}}
\newcommand{\tr}{{\textsf{T}}}
\newcommand{\rt}{{\textbf{r}}}

\DeclareMathOperator{\trace}{Trace}


\newcommand{\argmin}{ \operatornamewithlimits{argmin} }
\newcommand{\argmax}{ \operatornamewithlimits{argmax} }




\def\biz{\begin{itemize} }
\def\bizp{\begin{itemize}[<+->] }
\def\eiz{\end{itemize}}


\def\bfm{\begin{frame}}
\def\efm{\end{frame}}

\def\bena{\begin{enumerate}[<+-| alert@+>]}
\def\ben{\begin{enumerate}}
\def\een{\end{enumerate}}


\def\bbk{\begin{block} }
\def\ebk{\end{block}}






\makeatletter
%%%%%%%%%%%%%%%%%%%%%%%%%%%%%% Textclass specific LaTeX commands.
 % this default might be overridden by plain title style

%%%%%%%%%%%%%%%%%%%%%%%%%%%%%% User specified LaTeX commands.
%\usetheme{Warsaw}
\usetheme{Boadilla}
% or ...



%\setbeamertemplate{footline}[text line]{} % makes the footer EMPTY
%\setbeamertemplate{footline}[page number]{} % makes the footer EMPTY

%\usecolortheme{orchid} %not use is better 

\setbeamertemplate{footline}[text line]{%
  \parbox{\linewidth}{\vspace*{-2pt}Xiang Zhou\hfill CityU\hfill \insertpagenumber}}
%\setbeamertemplate{navigation symbols}{}

%\setbeamercovered{transparent}
% or whatever (possibly just delete it)


%\usepackage{babel}
\makeatother



 %
%\addtobeamertemplate{frametitle}{}{%
%\begin{tikzpicture}[remember picture,overlay]
%\node[anchor=south east,yshift=2pt] at (current page.south east) {\includegraphics[height=0.6cm]{CityU_Logo_Basic_Signature.eps}};
%\end{tikzpicture}}
%

\beamerdefaultoverlayspecification{<+->}
%the presentation acts as though a \pause command has been inserted between every two bullets, without the actual need to write \pause after each item.



\title{Introduction to Statistical  Machine Learning}

\author{\includegraphics[height=1.1cm,width=2.2cm]{../CityU_Logo_Basic_Signature.eps}
\\ $\ $ \\
Xiang Zhou  \\ $\ $ \\
}
\institute[]{  School of Data Science 
\\
 Department of Mathematics
\\
City University of Hong Kong
\\
~~
\\
\textup{  }
}

\date[]{}



\begin{document}
 
 


\maketitle
 


\frame{
{Mathematical Description of Data}

Notations
\biz
\item  input - output relation 
\begin{itemize}
\item $x$: inputs, feature vectors, predictors, independent variables.
\item $y$: output, response, dependent variable.
\end{itemize}

\item $\mathcal{X}$ and $\mathcal{Y}$ denote the spaces of the generic $x$ and $y$ variables, respectively.

\begin{itemize}
\item Generally $\Xcal = { \Real }^p$; qualitative features are coded using, for example, dummy variables
(such as 0,1, -1, etc).
\item Typically $\Ycal \in \Real^1$ is a scalar,  or takes a finite number of values as a subset of $ \NN $ ;
it can be a  vector in some scenarios.
\end{itemize}
\item The random variable $(X,Y)$ has the joint distribution $p(x,y)$
on the sample space $\Xcal \times \Ycal$.
\eiz
}

\frame{{Ground truth}
\biz
\item 
It is usually assumed that the ground truth for the relation between from input to the output  is a { deterministic} input-output mapping from $x\in \Xcal$  to 
$y_{\text{true}}\in \Ycal$:
\[ y_{\text{true}} = f^\star (x)\]
where the ground truth  $f^*$ is an unknown function and  has to be 
approximated by learning from  the training dataset.
\eiz
}


\frame{{Data as iid r.v. }
\biz
\item In {\bf supervised learning}, the data ( observations )  are given as the collection of the pairs \footnote{sometimes it is denoted as $\Dcal=\set{(x_i, y_i):  1\leq i \leq n}$ if the subindex has no ambiguity. }

$$\Dcal=\set{(x^{(i)}, y^{(i)}):  1\leq i \leq N} \subset (\Xcal\times \Ycal)^N$$
which  is  {assumed} iid samples of the r.v. $(X,Y)$ with an unknown joint distribution  $p(x,y)$ on the product space $\Xcal \times \Ycal$.
\biz
\item \hb{Regression}:   $\Ycal$ is continuous/quantitiative,  e.g.,  $\Real^{d}$ or its subset.
\item \hb{Classification}:  $\Ycal$ is discrete and finite (categorical variable), encoded by a finite number 
$\set{1,\ldots,K}$.  In this case, ``$y$'' is usually called ``label''.
\eiz 


\item In {\bf unsupervised learning}, the observations only have $\set{x^{(i)}}$, the information $y^{(i)}$
is zero or there is no definition of $y$ variable. The purpose is to identify the pattern of $\set{x^{(i)}}$ itself, such as 
model reduction.

%\item {\bf Reinforcement  learning}: the interaction between the agent and environment generates data for decision.
%There are no given observations.

\eiz
}
\frame{
{Output as perturbed   truth}
\biz
\item The observed $x^{(i)}\in \Xcal$ are samples from 
the marginal distribution $p_X$, i.e., $x^{(i)}\sim X$; in some cases,
they are deterministic and assigned by a procedure of experiment design.
\item  The  corresponding observed $y^{(i)}$ are assumed to be 
the {\it perturbed} truth $f^\star (x^{(i)})$   with additional measurement error $\eps^{(i)}$ which are assumed to be iid and independent from $X$.
 $$y^{(i)}=f^\star (x^{(i)}) + \eps^{(i)}.$$
 $\set{\eps^{(i)}}$ are assumed iid  and distributed as a generic r.v. $  \eps$.
\item The effect of the noise $\eps$ can never be eliminated by any statistical learning algorithms (\hb{irreducible error}).
\eiz
}

\frame{
\biz
\item  So, the joint distribution $p(x,y)$ of $(X,Y)$
is  completely determined by the triplet:
\[ ( p_X, f^\star,  p_{\eps} ) \] 
\biz
\item $p_X$:  the distribution of the input
\item  $f^\star$: the input-output function,
\item $p_\eps$: the distribution of error.
\eiz
\item The joint distribution $p(x,y)$ manifests by the available dataset $\Dcal$.
Given $X, Y$ random variables $\sim p$, how to identify $f^\star$?
\eiz
}


\frame{
\frametitle{Statistics and machine learning}

Different views and terminologies:
\begin{table} \centering
\begin{small}
\begin{tabular}{l|l}
\hline
\bf{Machine Learning} & \bf{Statistics} \\
\hline
Supervised learning & Classification/regression \\
Unsupervised learning & Clustering \\
Semisupervised learning & Classification/regression with missing responses \\
\hline
Features/outcomes & Covariates/responses \\
Training set/testing set & Sample/population \\
Learner & Statistical model  \\
Generalization error & Misclassification error/prediction error \\
\hline
\end{tabular}
\end{small}
\end{table}

}


\frame{
{Suggestion}

{Learn machine learning like a statistician or an applied mathematician, 
not a software engineer
}
\biz
\item Start from stating a problem,
not show an algorithm first:
Many times,  the students 
think  the methods/algorithms/procedures
as the problem itself.
\item Pay close attentions to the ``modelling'' process: 
how to turn the data problem into the statistical model.
In particularly the underlying principle which applies very  general.
\item Try to rigorously ( and in most general context)  understand (like a math theorem)
the heuristic arguments used  in practice, even for some toy examples.
\item Draw strict boundaries  between
general principles and specific 
methods.  In between,
computational method survives and thrives. 
\item 
Diagnose   and rationalize your results of numerical experiments.
Test different dataset/parameters/methods. 
\eiz

}


\frame{


{\color{blue} \huge \bf
{   Learning Theory:  An Approximation  Theory  Viewpoint}

}

\bigskip
Reference: 
\begin{center}
 Learning Theory: An Approximation Theory Viewpoint;
by F. Cucker and  D.X. Zhou, 
 Cambridge University Press 2007
\end{center}


\bigskip
\noindent\rule{\textwidth}{.5pt}

Given r.v.s $X $ and $Y$, find a function 
$f: \Xcal\to \Ycal$ so that 
$f(X)$ can explain $Y$ best in certain sense. 

}

%--------
\frame{
{Conditional Expectation as Optimal Prediction}

The best $L^2$ approximation of a function $f$
of the  r.v. $X$
to a r.v. $Y$ is achieved by the conditional probability.
The {\bf (generalized)    squared error}\footnote{also named as
generalization risk, mean square error,  $L_2$ error, etc.  }
 \begin{equation}
{\mathcal{E}(f):=\e   |Y-f(X)|^2  } 
\end{equation}
has a minimum  at 
 \[
\otherbox{ f^*(x)=\e(Y|X=x)}.\]
 i.e., 
\[ \e   |Y-f^*(X)|^2= \min_{f \mbox{\footnotesize: a Borel   function }}  \e (  |Y-f(X)|^2    ) \]

\vskip 16mm
\footnotesize{$f$ in fact should satisfy the 
condition that the RHS $L_2$ norm exists.}
}

%%%--------------------------
\frame{
\begin{proof}
\biz
\item 
We show first that   $\e[(Y-f^*(X) )  h(X) ]=0$
\footnote{
sometimes it is denoted 
$Y-f^*(X) \perp h(X)$,
the perpendicular property in $L_2$ space.
}
 is true  for any function $h$.
Using the double expectation theorem
\footnote{$\e [\e( Y|X )] = \e Y$}, we have 
\[
\begin{split}
&\e[(Y- f^*(X) )  h(X) ] = \e \left[ \e[Y-f^*(X) ) | X]  h(X)    \right ]\\
=& \e \left[ \e(Yh(X)|X) -  f^*(X) h(X) \right] =0.
\end{split}
\]
\item 
 Note that 
$(y-f(x))^2=(y-f^*(x))^2+(f(x)-f^*(x))^2-2(y-f^*(x))h(x)$ where $h(x)=f(x)-f^*(x)$,
then for any $f$
\begin{equation}
 \label{proj-cond-exp-3}
\boxed{
\e (  |f(X)-Y|^2    )
=\e (  |f^*(X)-Y|^2   ) + \e\left[ \abs{f(X)-f^*(X)} ^2 \right]
}
\end{equation}
 \eiz 
\end{proof}

}


\frame{
\begin{center}
\includegraphics[width=0.8\textwidth]{optimalL2cond.png}
\end{center}


\noindent\rule{\textwidth}{.5pt}

reference for elementary math:
\href{http://www.ee.nthu.edu.tw/cschang/Talk01142008.pdf}
{Understanding Conditional Expectation via Vector
Projection}
}

\frame{

The following exercise is to use the viewpoint of variational calculus:
minimizing functions in the  function space

\begin{ex}
Use the method of Calculus of Variation to solve 
\footnote{Rigorously, $f$ is in the $p$-weighted $L_2$
space}

\[
\inf_{f} \iint (f(x)-y)^2 p_{X,Y}(x,y) dx dy
\]
where $p_{X,Y}$ is the joint pdf of the r.v.s $(X, Y)$.
The optimal $f^*$ satisfies 
$$ \int (f^*(x)-y)p_{X,Y}(x,y)dy=0,~~
\quad \forall x$$
i.e., $\e Y = \e f^*(X)$
\end{ex}
What if generalized  the $L_2$ norm to $L_p $ norm ?
\bigskip
}

\frame{
\begin{ex}[Conditional distribution of multivariate Gaussian r.v.]

Suppose that $X=(X_1, X_2)$ is a two dimensional Gaussian random variable with mean $\mu=(\mu_1, \mu_2) $ and the covariance matrix $\Sigma=\left(
  \begin{array}{cc}
    \sigma_1^2 & \rho\sigma_1\sigma_2  \\
    \rho\sigma_1\sigma_2 & \sigma_2^2 \\
  \end{array}
\right) $. What is the conditional pdf $p(x_1|x_2)$ of $X_1$ given $X_2=x_2$? 
What is the expectation of $X_1$ given $X_2=x_2$?
What is the 
\href{https://en.wikipedia.org/wiki/Multivariate_normal_distribution\#Conditional_distributions}{general result} in $n$ dimension ?
\end{ex}

\[
X_1|_{X_2=x_2}\sim \mathcal{N}(\mu_1+\rho\frac{\sigma_1}{\sigma_2}(x_2-\mu_2), \sigma_1^2(1-\rho^2))
\]

}
\frame{
\biz
\item The objective function $\mathcal{E}(f)$ is also called 
loss function, risk, etc.
 \item $\e f^*(X) =\e Y$:  $f^*(X)$ is an unbiased estimate of $Y$;
\item
The variance of the difference between $Y$ and the optimal prediction $f^*(X)$
is
$$\sigma^2_*(x):= \e  \left[(Y-f^*(X))^2 \vert X=x\right]$$ 
  \item
  Take average of $\sigma^2(x)$ over $x$, then
the averaged uncertainty is 
 \[ \sigma^2_* := \e_X  \sigma^2(X)=\e  \left[\abs{Y-f^*(X)}^2 \right]=\mathcal{E}(f^*)
 \]
is the   variance  of the measurement error $Y-f^*(X)$:
\hb{irreducible error}
\eiz
}

\frame 
{{Notations}
We have shown that \eqref{proj-cond-exp-3}
for any two r.v. $X, Y$:
\begin{equation}
 \begin{split}
\Ecal(f)= \underbrace{
\e(  |f(X)-Y|^2    )}
_{\text{\bf Mean Square Error }}
 & = 
 \underbrace{\e_{X,Y} (  |f^*(X)-Y|^2   )}
 _{\Ecal(f^*), {\text{\bf Bayes error} } } + 
 \underbrace{
 \e_X\left[ \abs{f(X)-f^*(X)} ^2 \right]
 }
 _{\text{\bf model error} } 
\end{split}
\end{equation}
where $f^*(x)=\e(Y \vert X=x)$ is called {\bf Bayes rule}.\biz
\item Bayes error:  irreducible error;
\item Model error:  the distance from $f$ to $f^*$.
\eiz
}

\frame {
\frametitle{ Application to classification with 0-1 loss}

\begin{itemize}[<+->]
\item Assume $Y \in \{1,\ldots,K\}$ and $X \in {\Real}^p$.
\item $f:\Xcal \to \Ycal$ is a piece-wise constant function.
\item A Loss function $\ell(Y, f(Y))$ for penalizing errors in misclassification.
\item Most Common choice is the 0-1 loss
$$
\ell (Y , f(X))=I(Y \neq f(X))
=\begin{cases}
1 & \mbox{ if } Y \neq f(X)
\\
0 &   \mbox{ if }   Y = f(X)
\end{cases}.
$$
\item The expected prediction error, or the generalization error, is
$$
\Ecal (f)=\e I(Y \neq f(X))=\p (Y \neq f(X))=1-\p(Y=f(X)).
$$
\item The Bayes rule minimizing $\Ecal(f)$ is
the one maximizing $\p(Y=f(X))=
\int_\Xcal  \p(Y=f(x) \vert X=x) p_{X}(x)dx$,
which is 
$$
f^*(x)=\argmax_k \p(Y=k|X=x).
$$
since  $\p(Y=k|X=x) \leq \p(Y=f^*(x) \vert X=x ), \forall k $
\end{itemize}

}



\frame{
{How do we estimate $f^*$ ? }
\biz
\item direction method by definition of conditional expectation; 
\item optimization method by minimization of generalization error;
\eiz

}
\frame {

\frametitle{Nearest neighbors method to approximate  condition expectation}

A natural way to approximate the function $\e (Y \vert X=x)$ is to 
replace the expectation by the average:
$$f^*(x)=\e (Y\vert X=x) \approx \mbox{AVE}\set{y_i :  \mbox{ where } ~x_i=x},
~~\forall x\in \Xcal$$
or
$$f^*(x)=\int_\Ycal y\, p_{Y|X}(y;x ) dy $$ 
where $p_{Y|X}(y;x )
=\frac{p_{X,Y} (x,y)}{p_X(x)}$
by estimating the density $p_{X,Y}(x,y)$ first?

Question: Would these work well in practice ? in high dimension ?

\pause

(1) this formula is defined point-wisely;
(2) one does not have an accurate estimation of the expectation or the joint distribution $p(x,y)$,
particularly in high dimension.
}
%%% ------
\frame{
{$k$-NN}

\biz 
\item Use windows with size $\Delta$
\[
\e (Y\vert X=x) \approx \mbox{AVE}\set{y_i :  \mbox{ where } ~
\abs{x_i-x} \leq \Delta}
\]

\item Use $k$ -nearest neighbors:
have a look at its neighbors,
and take a vote:
$$
\e (Y\vert X=x) \approx \mbox{AVE} \set{y_i :  x_i   \in {\cal N}_k(x) }
$$
where ${\cal N}_k(x)$ is a neighborhood of $x$ that contains exactly $k$ neighbors
(k-nearest neighbors).

\eiz 
}

\frame {

\frametitle{Curse of dimensionality}

k-NN can fail in high dimensions, because it becomes difficult to gather $k$ observations close to a target point $x_0$.
\pause
\begin{itemize}[<+->]
\item Neighborhoods tend to be spatially large, and estimates are
biased.
\item Reducing the spatial size of the neighborhood means reducing $k$, and
the variance of the estimate increases.
\end{itemize}

\pause
In general (see Figure 2.6),
\begin{itemize}[<+->]
\item Most points are at the boundary, and points tend to be equidistant (Hall et al., JRSS-B, 2005).
\item Sampling density is proportional to $n^{1/p}$: the number of sample size increases exponentially in dimension $p$: If 100 points are sufficient
to estimate a function in ${\Real}^1$, $100^{10}$ are needed to achieve similar
accuracy in ${\Real}^{10}$.
\end{itemize}
This is similar to the curse of dimensionality in using the mesh grid based method to solve
high dimensional PDE.

}





\frame{

{\color{blue} \huge 

Minimizing the generalization error: learning as minimization}

}
\frame{
\biz
\item Loss function $\ell(y, \hat{y}):  \Ycal  \times \Ycal \to \Real^+\cup\set{0}$
\biz
\item  $L_2$, $L_1$ norm
\item  0-1 loss:  $\ell(y,\hat{y})= I( y\neq \hat{y})$.
\item  ...
\eiz
\item Population loss(Generalization error): $\Ecal (f) = \e \ell (Y, f(X))$
\eiz



\par
The direct  approach  of minimizing the risk \[ \min_{f} \mathcal{E}(f)=
 \e \ell (Y, f(X))\]
where $f$ can be in a very general class of functions, such as
continuous function on $\Xcal$, or even piecewise continuous function.

It is important to specify a  hypothesis space $\mathcal{H}$ to restrict the search space.

\ben

\item 
the function space of $f$  to search:  \hb{ hypothesis space(model class)} $\mathcal{H}$.



\item approximate the expectation $\e$ in $\mathcal{E}$ with  the use of data $\Dcal=\set{x_i, y_i}$.
\een


}
\frame{
%{hypothesis space and training error}
{Notations}
\ben

\item 
 
 The minimizer  of $$\min_{f\in \mathcal{H}}\mathcal{E}(f)$$
is denoted by $f_{\mathcal{H}}$. 
 

\item  
Population risk is approximated by 
empirical risk associated with $\Dcal=\set{(x_i, y_i)}: i=1,\ldots,N$:
\[
\mathcal{E}(f) = \e \left [ \abs{Y-f(X)}^2\right]
\approx 
\mathcal{E}_N(f)=\frac{1}{N} \sum_ i \ell (y_i, f(x_i))  
\]
The learning algorithm produces the  learnt function  $ \hat{f}_\Dcal $
(\hb{ regression function})
as the  solution to  $$
\Ecal_N ( \hat{f}_\Dcal ) = \min_{f\in \Hcal}\mathcal{E}_N(f)=
\min_{f\in \Hcal} \frac{1}{N} \sum_{(x_i,y_i)\in \Dcal}\ell(y_i,f(x_i))$$
\biz
\item 
The {\bf training error} is   $\mathcal{E}_N(\hat{f}_\Dcal)$ .
\item
  The {\bf test error} is   $\mathcal{E}(\hat{f}_\Dcal)$ .
\eiz
\een


}

\frame{


\biz

\item
$\hat{f}_\Dcal$ depends on {\it three} elements:
 the hypothesis space $\Hcal$,  the (training) data $\Dcal$ ( random ! ) and  the loss function $\ell$.
So, $\hat{f}_\Dcal$ is a random function to approximate the ground truth $f$.
\item
For the fixed $\Hcal$ and $\ell$,  $\hat{f}_\Dcal$ is essentially a mapping from $\Dcal$ to the function space
$\Hcal$.
 As $N\to \infty$, $\hat{f}_\Dcal \to  f_{\Hcal}$ by the law of large number.
\eiz
}
 
\frame{
{Hypothesis space: representation of the function $f$}
\biz
\item parametric approach:  e.g. $f(x)=\theta\cdot x$ with a set of finite and (usually fixed) number of parameters
$\theta$
\item non-parametric approach: functional analysis viewpoint, $f$ is only in some function space.
(such as in the traditional computational math for representing the solution to some PDE )
\item there is no rigorous boundary between parametric/non-parametric: eventually 
the computer represents a function $f$ only by a finite number of freedoms. 
\item need to be restricted to certain model class in practice. Examples:   linear model,  polynomial, spline, kernel machine,   neural network, etc.
\eiz 
}



\frame {

\frametitle{some examples of $\Hcal$}

\begin{itemize}[<+->]
\item Linear function (chapter 3 , ESL)
\item Basis functions and dictionary methods (chapter 5, ESL)
$$
f_{\theta}(x) = \sum_{m=1}^M \theta_m h_m(x),
$$
where $h_m$'s are pre-specified basis functions.
\item Kernel methods and local regression (chapter 6, ESL)
$$
RSS(f, x_0) = \sum_{i=1}^n K(x_0,x_i) (y_i-f(x_i))^2,
$$
where $K$ is a kernel measuring the closeness between points.
\item Roughness penalty (chapter 5, ESL)
$$
\mbox{Penalized RSS}(f; \lambda) = \mbox{RSS}(f) + \lambda J(f),
$$
where $J(f)$ is a regularization term on the model complexity.
\end{itemize}

}

\frame{
\underline{The generalization  error   is  then decomposed into} 
\bigskip
\[
\otherbox
{
\Ecal(f^*) = 
\underbrace{ {\Ecal(f^*) - \Ecal(f_\Hcal) }}_\text{\bf approximation error}
+ \underbrace{{\Ecal(f_\Hcal)- \Ecal_N(\hat{f}_\Dcal)} }_\text{\bf sampling error}
+  
\underbrace{\Ecal_N(\hat{f}_\Dcal)}_\text{\bf training error}
}
\]


}


\frame{

\biz
\item 
Approximation error:
characterized how large (complex)
the hypothesis space $\mathcal{H}$ is.
analogy to the interpolation 
inequality in finite element method.
A larger space of $\Hcal $ indicates
\biz
\item
A  larger effective dimension of this space;
\item A smaller approximation error;
lower bias
\item More difficult for numerical optimization;  
\eiz

\item 
Sampling error:
vanishes if the number of data
points $N$ goes to infinity. This error may be analyzed by
probability inequality 
(e.g. Chebychev);
analysis method used for 
Monte Carlo  simulation.
A larger sample size for training data indicates:
\biz
\item Smaller sampling error; lower variance; 
\item more challenges for data collection and  Big Data techniques;
\eiz
\item  
Training error: 
\biz
\item design of optimization method, numerical convergence rate,
\item  the time and memory cost ... 
\item the output is  a further approximation to $\hat{f}_\Dcal$:
the optimization algorithm stops with a time cost $T$:
$\hat{f}_{\Dcal}^{(T)}$
\eiz 
 \eiz
 }
 
 %%%--------------------------------------
\frame{
{Very Important Remarks }

\biz
\item Although different fields focus on
each of three errors separately,
to address the problems for machine learning 
needs  trans-disciplinary efforts:
\biz
\item approximation theory
\item statistical learning
\item computation  
\eiz
A   joint strategy  is very important to have
a complete picture of the machine learning problem.

\item  These three topics are not completely separate.
 For example, the necessary number of sample size $N$
 may depend on the complexity of hypothesis space,
 which further depends  on the  dimension of $\Xcal,\Ycal$
 and the specific representation form of functions in $\Hcal$. 
\eiz
}


\frame{{Generalization Gap}
The optimal function in theory is $f^*$; 
the  optimal function in practical computation is 
$\hat{f}_\Dcal$. The difference in  risks 
of these two:
\[
 \Ecal(\hat{f}_\Dcal) - 
\Ecal(f^*)
\]
is called {\bf generalization gap}:
it measures to which extent 
the performance of the algorithm
learnt from a given training data set $\Dcal$
is valid for the whole distribution $p(x,y)$.
}
\frame{
{First inequality for Generalization Gap}
\begin{ex}
Prove the upper bound of the generalization gap: 
 \[
 \begin{split}
 \otherbox{
 \Ecal(\hat{f}_\Dcal) 
 - \Ecal(f^*)
% =
% \Ecal(\hat{f}_\Dcal) - 
%\Ecal_N(\hat{f}_\Dcal)
%+
%\Ecal_N (\hat{f}_\Dcal)
%-
%\Ecal_N (f^*)
%+\Ecal_N (f^*)
%- \Ecal(f^*)
\leq 2 \sup_{f\in \Hcal} \abs{ \Ecal_N (f)- \Ecal(f)}
}
\\
\approx 2 \sup_{f\in \Hcal}  \frac{1}{\sqrt{N}} \Var (\ell(Y,f(X)))
\end{split}
 \]
  \end{ex}
 The first RHS depends on the hypothesis space
 $\Hcal$ and the dataset $\Dcal$ of size $N$ in $\Ecal_N$;
 and the second depends on $\Hcal$ and the joint distribution $p(x,y)$
 of $(X,Y)$.


The supremum in the above inequality
 can describe the {\bf complexity } of  hypothesis space $\Hcal$,
 which increases when $\Hcal$ is ``larger''.
 WARNING: this upper bound is not sharp.
 A good lower bound is extremely difficult to obtain.
}


\frame{

{\color{blue} \huge 
Statistical  Learning:
Viewpoint of Bias-Variance
Trade-off}


}
\frame{
{Bias-Variance Decomposition / Trade-off}

Let $f$ be a given deterministic function 
$\mathcal{X}\to\mathcal{Y}$.

Assume that the response r.v. 
$Y$ is defined by 
\[
Y=f(X)+\eps
\]
where  the 
r.v.  $\eps$ is independent of $X$
and has mean $0$ and variance $\sigma_\eps^2$.
Then 
$f^*(x)=\e[Y\vert X=x]=f(x)$ is the optimal approximation
in the square error sense.
}
\frame{ {test error decomposition}
Based on a {training} dataset $\Dcal=\set{x_i, y_i}$ where $y_i=f(x_i)+\eps_i$ and $\eps_i  \sim \eps$, one learns a regression  function, 
denoted by 
$\hat{f}_\Dcal(\cdot)$, then
the {\bf test error} 
\footnote{
also called {\bf Prediction Error}.
In some cases, this definition does 
not contain $\eps_0$ contribution.
}
at a new input $x_0$ 
(with a new independent measurement error $\eps_0\sim \eps$) is 
\begin{align*}
& \e [  ( Y-\hat{f}_\Dcal(X))^2
\vert X=x_0] 
=
 \e_{\eps_0, \Dcal} [  ( f(x_0)+\eps_0-\hat{f}_\Dcal(x_0))^2
] ~~\\
=&\sigma_\eps^2+ \e_\Dcal [  \left( f(x_0)-\hat{f}_\Dcal(x_0)\right)^2] 
~~\because   \e (\eps_0)=0 , \mbox{ and } \Dcal \perp \eps_0
\\
=&
\sigma_\eps^2+ \e_\Dcal [  \left( f(x_0)-
\e\hat{f}_\Dcal(x_0) + \e \hat{f}_\Dcal(x_0)- \hat{f}_\Dcal(x_0) \right)^2] \\
=&
\otherbox{
\sigma_\eps^2+ 
\underbrace{ ( f(x_0)-
\e \hat{f}_\Dcal(x_0))^2}
_\text {Bias$^2$}
 + 
 \underbrace{
 \Var (\hat{f}_\Dcal(x_0))
 } _\text{Variance}}
\end{align*}
 Sometimes, this quantity should be taken expectation w.r.t. 
 $x_0 \sim p_X$.
 }
 
 
\frame{
%For linear regression:
%\[ \Var (\hat{f}_\Dcal(x_0))
%\sim p\sigma^2_\eps/N  \]
%where $p$ is the number of components,
%i.e.,  the dimension 
%of $X$. $\mathcal{X}=\Real^p$.
%

\biz
\item 
low bias:  large model space, 
low training error, 
overfitting, bad generalization ability
(high variance);
 \item low variance:  
rigid model space,
insensitive to the perturbation of the dataset used in fitting;
good generalization for the new data from the same distribution. 
\item BAD news:
it is almost impossible to decrease the bias and variance terms
simultaneously!

 \item Criteria for model assessment or variable selection:
{
\color{blue}\bf good trade-off between the bias and variance
}

\biz
\item Analyse the bias or variance 
or model complexity to have analytical 
results (very limited cases)
\item  
A practical method is to estimate $\e [  ( Y-\hat{f}_\Dcal(X))^2]
\approx \frac{1}{n} \sum_j (Y'_j-\hat{f}_\Dcal(X'_j))$
where $X'_j$ are not from $\Dcal$
but an independent dataset $\Dcal'$.
\item Where is the ``new'' dataset $\Dcal'$?
cross-validation
(split 1 cent into 1/2 cents)  and bootstrap
(1 cent used twice ) .

\eiz


\eiz
}

%%%------------------------------------------------------
\frame{
\begin{center}
\begin{figure}
\includegraphics[width=\textwidth]{bias-variance-tradeoff.png}
\caption{  Bias-Variance Tradeoff as a Function of Model Capacity/Complexity
}
\end{figure}
\end{center}
}
%%%------------------------------------------------------


\frame {

\frametitle{Model assessment}

Typical objectives:
\begin{itemize}
\item Choose a value of a tuning parameter  (\hb{hyperparameter})
used in the model
(such as $k$ in k-NN)
\item Estimate the prediction performance  (test error) of a given model
\end{itemize}

\vspace{3mm}
Remarks:
\begin{itemize}
\item For both objectives, the best approach is to run the procedure on
an independent test set, if one is available.
\item If possible, one should use different test data for (1) and (2) above: a
{\it validation set} for (1) and a {\it test set} for (2).
\item Often there is insufficient data to create a separate validation or test set. In this case, {\it Cross-Validation} is useful.
\end{itemize}

}



%%%--------------------------------------------------------------------

\frame {

\frametitle{K-fold cross validation}
Denote the hyper-parameter by $\lambda$.
K-fold cross validation is the most popular method for estimating a tuning parameter $\lambda$. 

\vspace{3mm}
Divide the dataset (of size $N$) into $K$ subsets: ${\cal A}_1,\ldots,{\cal A}_K$ ($K=2,5,10$ or $N$)
\begin{itemize}
\item For each $k = 1, \ldots, K$, fit the model with parameter $\lambda$ to $\{{\cal A}_1,\ldots,{\cal A}_{k-1},{\cal A}_{k+1},\ldots,{\cal A}_K\}$ giving $ {f}^{-k}_\lambda(\cdot)$, and compute its prediction error on ${\cal A}_k$: $$E_k(\lambda)=\sum_{x_i \in {\cal A}_k} \ell (y_i, f^{-k}_\lambda(x_i)).$$
\item The average of these $K$ values $E_k(\lambda)$ give  the cross-validation error (per sample)
    $$
    CV(\lambda):=\frac{1}{N} \sum_{k=1}^K E_k(\lambda).
    $$
\item Choose the optimal $\lambda^*$ yielding the smallest $CV(\lambda)$.
\end{itemize}

}



%%%--------------------------------------------------------------------

\frame {

\frametitle{K-fold cross validation}

\begin{itemize}
\item Cross-validation is often abbreviated as CV.
\item In the subset selection procedure, $\lambda$ is the subset size
\item $f^{-k}(\lambda)$ is the  best model of size $\lambda$, found from the training set that leaves out the $k$-th part of the data
\item $E_k(\lambda)$ is its estimated test error on the $k$-th part.
\item Using $K$-fold CV, the $K$ test error estimates are averaged to give the final CV estimated test error.
\item The output is the model associated with  $\lambda^*$,
typically, 
computed  by using all $N$ data. 
\end{itemize}

}



%%%--------------------------------------------------------------------

\frame {

\frametitle{Bootstrap}

\begin{itemize}
\item Bootstrap works by sampling $N$ times with replacement from the training set to form a ``bootstrap" data set. Then model is estimated on the bootstrap data set,
and predictions are made on the original training set.
\item This process is repeated many times and the results are averaged.
\item {\it Bootstrap is most useful for estimating standard errors of predictions.}
\item Can also use modified versions of the bootstrap to estimate prediction error. Sometimes produces better estimates than CV (still an open question!)
\end{itemize}

}





\end{document}


