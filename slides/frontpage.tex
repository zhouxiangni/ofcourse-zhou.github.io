




 
%\usepackage{mathptmx}
%\renewcommand{\sfdefault}{lmss}
\usepackage[T1]{fontenc}
%\usepackage[latin9]{inputenc}
\usepackage[utf8]{inputenc}

\synctex=-1

\usefonttheme{professionalfonts}

%\setbeamertemplate{navigation symbols}{}
%\setbeamertemplate{caption}[numbered]


\useinnertheme{rectangles}
%http://tex.stackexchange.com/questions/11168/change-bullet-style-formatting-in-beamer

 \AtBeginDocument{
  \addtolength\abovedisplayskip{-0.4\baselineskip}%
  \addtolength\belowdisplayskip{-0.4\baselineskip}%
}%change the space between text lines and the math formula


\usepackage{pifont}
%Postscript ZipfDingbats font
%the command \ding{number}, will print the specified symbol

\usepackage{fontawesome}
%icon package
\DeclareFontFamily{U}{FontAwesomeOne}{}
\DeclareFontShape{U}{FontAwesomeOne}{m}{n}{<-> FontAwesome--fontawesomeone}{}
\DeclareRobustCommand\FAone{\fontencoding{U}\fontfamily{FontAwesomeOne}\fontseries{m}\fontshape{n}\selectfont}
\DeclareFontFamily{U}{FontAwesomeTwo}{}
\DeclareFontShape{U}{FontAwesomeTwo}{m}{n}{<-> FontAwesome--fontawesometwo}{}
\DeclareRobustCommand\FAtwo{\fontencoding{U}\fontfamily{FontAwesomeTwo}\fontseries{m}\fontshape{n}\selectfont}
\DeclareFontFamily{U}{FontAwesomeThree}{}
\DeclareFontShape{U}{FontAwesomeThree}{m}{n}{<-> FontAwesome--fontawesomethree}{}
\DeclareRobustCommand\FAthree{\fontencoding{U}\fontfamily{FontAwesomeThree}\fontseries{m}\fontshape{n}\selectfont}

%ftp://ftp.dante.de/tex-archive/fonts/fontawesome/doc/fontawesome.pdf
%http://tug.ctan.org/info/symbols/comprehensive/symbols-a4.pdf


\usepackage{amsmath,amssymb,amsfonts,bm,mathrsfs,mathtools}

\usepackage{tikzsymbols}
%\usepackage[tikz]{bclogo}



\usepackage{perpage}
\MakePerPage{footnote} %reset for each page
%\renewcommand{\thefootnote}{\fnsymbol{footnote}} %use symbol, limit less than 9 symbols



%%%% HIGHTLIGHT  and annotation &=%%%%%%%%
\usepackage{color,xcolor}
 \usepackage{todonotes}

\usepackage[normalem]{ulem}

\usepackage[many]{tcolorbox}

\tcbset{fonttitle=\scriptsize}
\tcbset{highlight math style={enhanced,
  colframe=red!40!black,colback=yellow!20!white,arc=2pt,boxrule=.2pt,
  }}
  \newtcbox{\otherbox}[1][]{nobeforeafter,math upper,tcbox raise base,
enhanced,frame hidden,boxrule=0pt,interior style={top color=green!10!white,
bottom color=green!10!white,middle color=green!50!yellow},
fuzzy halo=1pt with green,#1}
%%\tcbhighmath{math here}
%% \otherbox{math here}



%%%%% HIGHLIGHT %%%%%%
\newcommand{\hb}[1]{{\color{blue}{#1}}}
%\noindent\rule{\textwidth}{.5pt}

%:
\usepackage{soul}

\newcommand\hcancel[2][black]{\setbox0=\hbox{$#2$}%
\rlap{\raisebox{.45\ht0}{\textcolor{#1}{\rule{\wd0}{1pt}}}}#2}
%cross to delete

\newcommand{\mcb}[2]{\colorbox{#1}{$\displaystyle #2$}}
%highlight math

\newcommand{\hlfancy}[2]{\sethlcolor{#1}\hl{#2}}
%specified color , for\hl

\newcommand\myhl{\bgroup\markoverwith
  {\textcolor{yellow}{\rule[-.5ex]{2pt}{2.5ex}}}\ULon}



\mode<presentation>{ \usetheme{boxes} }

%write Matlab code
\usepackage{listings}
 \definecolor{dkgreen}{rgb}{0,0.6,0}
\definecolor{gray}{rgb}{0.5,0.5,0.5}
\definecolor{mauve}{rgb}{0.58,0,0.82}
\lstset{frame=tb,
  language=Matlab,
  aboveskip=3mm,
  belowskip=3mm,
  showstringspaces=false,
  columns=flexible,
  basicstyle={\small\ttfamily},
  numbers=none,
  numberstyle=\tiny\color{gray},
  keywordstyle=\color{blue},
  commentstyle=\color{dkgreen},
  stringstyle=\color{mauve},
  breaklines=true,
  breakatwhitespace=true
  tabsize=3
}

\usepackage[lastexercise]{exercise}

\newtheorem{ex}{Exercise}
\newtheorem{property}{Property}
\newtheorem{ag}{Algorithm}
\newtheorem{remark}{Remark}
\newtheorem{den}{definition}
\newtheorem{assumption}{Assumption}


\usepackage[nosolutionfiles]{answers}
\Newassociation{sol}{Solution}{ans}



\usepackage{empheq}
\usepackage{comment}
%\usepackage{lscape}
\usepackage{multirow}
\usepackage{url,hyperref}

\hypersetup{
 %   bookmarks=true,         % show bookmarks bar?
    unicode=false,          % non-Latin characters in Acrobat's bookmarks
    pdftoolbar=true,        % show Acrobat's toolbar?
    pdfmenubar=true,        % show Acrobat's menu?
    pdffitwindow=false,     % window fit to page when opened
    pdfstartview={FitH},    % fits the width of the page to the window
    pdftitle={My title},    % title
    pdfauthor={Author},     % author
    pdfsubject={Subject},   % subject of the document
    pdfcreator={Creator},   % creator of the document
    pdfproducer={Producer}, % producer of the document
    pdfkeywords={keyword1} {key2} {key3}, % list of keywords
    pdfnewwindow=true,      % links in new window
    colorlinks=true,       % false: boxed links; true: colored links
    linkcolor=red,          % color of internal links (change box color with linkbordercolor)
    citecolor=green,        % color of links to bibliography
    filecolor=magenta,      % color of file links
    urlcolor=cyan           % color of external links
}


\usepackage{subfigure,epsfig,graphicx,graphics}

\DeclareGraphicsRule{.tif}{png}{.png}{`convert #1 `dirname #1`/`basename #1 .tif`.png}
   \DeclareGraphicsExtensions{.pdf}




\newcommand{\hw}{ {\underline{\tt Homework }} }
\newcommand{\hws}{ {\underline{\tt Homework$\star$}} }
\newcommand{\optional}{ {\it optional} }

\newcommand{\MATLAB}{ \texttt{MATLAB}}
\newcommand{\python}{ \texttt{python}}
\newcommand{\Rlang}{ \texttt{R}}
\newcommand{\SAS}{ \texttt{SAS}}
\newcommand{\MC}{Markov Chain}


\newcommand{\tm}{transition matrix}
\newcommand{\rv}{random variable}
\newcommand{\spl} {supervised learning }
 

\newcommand{\dis}{\underline{\tt discussion}: }
\newcommand{\pri}{\underline{\tt principle}: }




\newcommand{\bq}{\scalebox{6}{\textbf{?} }}
\newcommand{\sq}{\scalebox{2}{\textbf{?} }}
\newcommand{\ck} {  {\scalebox{0.8} {\Interval}   } }

\newcommand{\eps}{\varepsilon}
\newcommand{\To}{\longrightarrow}

% 
\newcommand{\Dcal}{\mathtt{D}}
\newcommand{\Hcal}{\mathcal{H}}
\newcommand{\Ecal}{\mathcal{E}}
\newcommand{\Xcal}{\mathcal{X}}
\newcommand{\Ycal}{\mathcal{Y}}
\newcommand{\Zcal}{\mathcal{Z}}

%%Calculus 

\renewcommand{\d}{\ensuremath{\mathrm{d}}}
\newcommand{\dt}{ \ensuremath{\mathrm{d} t } }
\newcommand{\dx}{ \ensuremath{\mathrm{d} x} }
\newcommand{\dy}{ \ensuremath{\mathrm{d} y } }

%indicator function
\newcommand{\indf}{ \ensuremath{\mathbf{1} } }



%probability
\newcommand{\p}{ \mathbb{P}}
\newcommand{\prob}{{\Pr}}
\newcommand{\PP}{\mbox{PP}}%Poisson process
%condition prob
\newcommand{\cPr}[2]{{\Pr\left(#1\mid #2\right)}}

\newcommand{\FF}{{\mathbb{F}}}

\newcommand{\e}{ \operatorname{\mathbb E}}
\newcommand{\Var}{\operatorname{\mathbb{V} }}
\newcommand{\var}{\operatorname{\text{Var} }}
\newcommand{\MSE}{\operatorname{\text{MSE} }}

\newcommand{\Std}{\operatorname{std}}
\newcommand{\Cov}{\operatorname{cov}}

%Matrix  %mathbf
\newcommand{\Pb}{{\mathbf{P}}}
\newcommand{\Qb}{{\mathbf{Q}}}
\newcommand{\Mb}{{\mathbf{M}}}
\newcommand{\cb}{\mathbf{c}}
\newcommand{\bb}{{\mathbf{b}}}

\newcommand{\Tb}{\mathbf{T}}

\newcommand{\Wb}{\mathbf{W}}
\newcommand{\wb}{\mathbf{w}}
\newcommand{\Xb}{\mathbf{X}}

\newcommand{\xb}{\mathbf{x}}

\newcommand{\Wtn}{\mathbb{W}}
\newcommand{\btn}{\mathbf{b}}



\newcommand{\eye}{{\mathbf{I}}}
%identity matrix
\newcommand{\onem}{{\mathbb{1}}}
\newcommand{\idor}{\mathbf{1}}
\newcommand{\ii}{\mathbf{i}}
%imaginary symbol

\usepackage{tikz}

%State number
\newcommand{\snum}[1]{ \raisebox{.5pt}{\textcircled{\raisebox{-.9pt} {#1}}}}

 \usetikzlibrary{arrows}
\usetikzlibrary{shapes}

%\newcommand{\snum}[1]{%
 % \tikz[baseline=(char.base)]\node[anchor=south west, draw,rectangle, rounded corners, inner sep=1.4pt, minimum size=5mm,
   % text height=1.3mm](char){\ensuremath{#1}} ;}

\newcommand*\circled[1]{\tikz[baseline=(char.base)]{
            \node[shape=circle,draw,inner sep=.4pt] (char) {#1};}}


%real number
\newcommand{\Real}{{\mathbb{R}}}
%integer
\newcommand{\ZZ}{\mathbb{Z}}
%positive integer
\newcommand{\NN}{\mathbb{N}}



\newcommand{\inpd}[2]{\left\langle #1, #2 \right\rangle}
\newcommand{\abs}[1]{\left\vert#1\right\vert}
\newcommand{\norm}[1]{\left\|#1\right\|}
\newcommand{\wt}[1]{{\widetilde{#1}}}
\newcommand{\set}[1]{\left\{#1\right\}}
\newcommand{\partiald}[2]{  \frac{\partial #1 }{\partial #2}}



\newcommand{\ie}{{\it{i.e.}}}



\newcommand{\transpose}{\textsf{T}} % or, \intercal
\newcommand{\diag}{\textsf{diag}}
\newcommand{\tr}{{\textsf{T}}}
\newcommand{\rt}{{\textbf{r}}}

\DeclareMathOperator{\trace}{Trace}


\newcommand{\argmin}{ \operatornamewithlimits{argmin} }
\newcommand{\argmax}{ \operatornamewithlimits{argmax} }




\def\biz{\begin{itemize} }
\def\bizp{\begin{itemize}[<+->] }
\def\eiz{\end{itemize}}


\def\bfm{\begin{frame}}
\def\efm{\end{frame}}

\def\bena{\begin{enumerate}[<+-| alert@+>]}
\def\ben{\begin{enumerate}}
\def\een{\end{enumerate}}


\def\bbk{\begin{block} }
\def\ebk{\end{block}}






\makeatletter
%%%%%%%%%%%%%%%%%%%%%%%%%%%%%% Textclass specific LaTeX commands.
 % this default might be overridden by plain title style

%%%%%%%%%%%%%%%%%%%%%%%%%%%%%% User specified LaTeX commands.
%\usetheme{Warsaw}
\usetheme{Boadilla}
% or ...



%\setbeamertemplate{footline}[text line]{} % makes the footer EMPTY
%\setbeamertemplate{footline}[page number]{} % makes the footer EMPTY

%\usecolortheme{orchid} %not use is better 

\setbeamertemplate{footline}[text line]{%
  \parbox{\linewidth}{\vspace*{-2pt}Xiang Zhou\hfill CityU\hfill \insertpagenumber}}
%\setbeamertemplate{navigation symbols}{}

%\setbeamercovered{transparent}
% or whatever (possibly just delete it)


%\usepackage{babel}
\makeatother



 %
%\addtobeamertemplate{frametitle}{}{%
%\begin{tikzpicture}[remember picture,overlay]
%\node[anchor=south east,yshift=2pt] at (current page.south east) {\includegraphics[height=0.6cm]{CityU_Logo_Basic_Signature.eps}};
%\end{tikzpicture}}
%

\beamerdefaultoverlayspecification{<+->}
%the presentation acts as though a \pause command has been inserted between every two bullets, without the actual need to write \pause after each item.

